\documentclass[letterpaper,11pt]{article}

% --1. 删除或注释掉 \usepackage[english]{babel},改用 ctex 管理中文 --
\usepackage[UTF8]{ctex}

% --2. 显式关闭 ctex 默认的字体集并手动指定中文字体(Mac系统示例)--
\ctexset{
  fontset = none
}
\setCJKmainfont{PingFang SC}
\setCJKsansfont{PingFang SC}
\setCJKmonofont{PingFang SC}

% 其余非中文相关的包保留即可
\usepackage{latexsym}
\usepackage[empty]{fullpage}
\usepackage{titlesec}
\usepackage{marvosym}
\usepackage[usenames,dvipsnames]{color}
\usepackage{verbatim}
\usepackage{enumitem}
\usepackage[hidelinks]{hyperref}
\usepackage{fancyhdr}
\usepackage{tabularx}
% \input{glyphtounicode}

\pagestyle{fancy}
\fancyhf{} 
\fancyfoot{}
\renewcommand{\headrulewidth}{0pt}
\renewcommand{\footrulewidth}{0pt}

% 调整页边距
\addtolength{\oddsidemargin}{-0.5in}
\addtolength{\evensidemargin}{-0.5in}
\addtolength{\textwidth}{1in}
\addtolength{\topmargin}{-.5in}
\addtolength{\textheight}{1.0in}

\urlstyle{same}

\raggedbottom
\raggedright
\setlength{\tabcolsep}{0in}

% 章节格式
\titleformat{\section}{
  \vspace{-4pt}\scshape\raggedright\large
}{}{0em}{}[\color{black}\titlerule \vspace{-5pt}]

%-------------------------
% \pdfgentounicode=1

%-------------------------
% 自定义命令
\newcommand{\resumeItem}[1]{
  \item\small{
    {#1 \vspace{-5pt}}
  }
}

\newcommand{\resumeSubheading}[4]{
  \vspace{-2pt}\item
    \begin{tabular*}{0.97\textwidth}[t]{l@{\extracolsep{\fill}}r}
      \textbf{#1} & #2 \\
      \textit{\small#3} & \textit{\small #4} \\
    \end{tabular*}\vspace{-7pt}
}

\newcommand{\resumeSubSubheading}[2]{
    \item
    \begin{tabular*}{0.97\textwidth}{l@{\extracolsep{\fill}}r}
      \textit{\small#1} & \textit{\small #2} \\
    \end{tabular*}\vspace{-7pt}
}

\newcommand{\resumeProjectHeading}[2]{
    \item
    \begin{tabular*}{0.97\textwidth}{l@{\extracolsep{\fill}}r}
      \small#1 & #2 \\
    \end{tabular*}\vspace{-7pt}
}

\newcommand{\resumeSubItem}[1]{\resumeItem{#1}\vspace{-4pt}}

\renewcommand\labelitemii{$\vcenter{\hbox{\tiny$\bullet$}}$}

\newcommand{\resumeSubHeadingListStart}{\begin{itemize}[leftmargin=0.05in, label={}]}
\newcommand{\resumeSubHeadingListEnd}{\end{itemize}}
\newcommand{\resumeItemListStart}{\begin{itemize}}
\newcommand{\resumeItemListEnd}{\end{itemize}\vspace{-5pt}}

%-------------------------------------------
%%%%%%  RESUME STARTS HERE  %%%%%%%%%%%%%%%%%%%%%%%%%%%%

\begin{document}

\begin{center}
    \textbf{\Large \scshape 尹烁霖 Shuolin Yin} \\ \vspace{5pt}
    \small +1 (647) 674-6127 $|$
    \href{mailto:leo.yin@mail.utoronto.ca}{leo.yin@mail.utoronto.ca} \\
    \href{https://github.com/leoyin1127}{https://github.com/leoyin1127} $|$
    \href{https://www.linkedin.com/in/shuolinyin/}{https://www.linkedin.com/in/shuolinyin/}
\end{center}

\section{教育经历}
\resumeSubHeadingListStart
  \resumeSubheading
  {多伦多大学}{加拿大多伦多}
  {计算机工程应用科学与工程学士(含 PEY Co-op 实习项目)}{2023年9月 -- 2027年4月}
  \resumeItemListStart
    \resumeItem{获得多伦多大学工程学院入学奖学金}
    \resumeItem{获得 Edward S. Rogers Sr. 入学奖学金}
  \resumeItemListEnd
\resumeSubHeadingListEnd

\section{工作经历}
\resumeSubHeadingListStart

\resumeSubheading
  {中国科学院自动化研究所 (CASIA)}{中国北京}
  {研究助理}{2024年2月 -- 2024年9月}
  \resumeItemListStart
    \resumeItem{通过系统性文献调研和实验设计,研究多模态大模型(MLLM)幻觉(Hallucination)问题的解决方案,为 ECCV 2025 论文发表奠定方法论基础}
    \resumeItem{使用 Python 和自定义标注工具搭建自动化数据处理流水线,用于生成多模态大模型基准测试数据,数据处理效率提升 300\%,标注准确率提高 40\%}
    \resumeItem{每周进行技术汇报和撰写详细进度报告,提升团队间协作效率并促进研究成果的交流}
    \resumeItem{使用 PyTorch 在多种前沿多模态大模型上开展系统性实验,评估幻觉问题的减缓策略并识别关键改进点}
  \resumeItemListEnd

\resumeSubheading
  {VolunTrack 非营利组织}{加拿大多伦多}
  {主席/创始人}{2022年6月 -- 2024年9月}
  \resumeItemListStart
    \resumeItem{创办并规模化运营非营利组织,通过建立四级管理体系,将团队扩展至 40+ 名成员并与全球超过 100 家非营利组织建立合作关系}
    \resumeItem{基于 React Native 与 React.js 开发完整的志愿者管理平台(涵盖 Web 和移动端),月活跃用户数量超过 1000+}
    \resumeItem{争取到 Google Cloud Platform 的合作,获得云资源补贴以及技术培训项目,培养团队中 20+ 名开发人员}
    \resumeItem{使用敏捷开发方法并结合 CI/CD 流程,在两年内成功部署 15+ 个主要功能、4 次重大版本迭代,部署效率提升 70\%}
    \resumeItem{使用 Firebase Realtime Database 和 Cloud Firestore 构建可扩展数据库架构,并集成 GCP 服务,为 100+ 个组织资料提供高效数据管理}
    \resumeItem{建立面向学生的技术指导项目,将其与真实世界项目对接,完成 30+ 个成功项目并在 10+ 个会议上进行演讲分享}
  \resumeItemListEnd 

\resumeSubHeadingListEnd

\section{项目 / 校园经历}
\resumeSubHeadingListStart

\resumeSubheading
  {IEEE UofT}{加拿大多伦多}
  {Web 团队成员}{2024年6月 -- 至今}
  \resumeItemListStart
    \resumeItem{使用 React.js(前端)与 Django(后端)构建并维护全栈 Web 应用,包括 IEEE UofT 官方网站和活动平台,同时实现响应式设计}
    \resumeItem{通过优化 API 和缓存策略提升网站性能,改善加载速度和用户体验}
    \resumeItem{与团队协作开发新功能,维护已有功能,确保跨平台的一致用户体验}
  \resumeItemListEnd

\resumeSubheading
  {YiXing 软件开发}{加拿大多伦多}
  {项目负责人 / 联合创始人}{2023年6月 -- 2024年5月}
  \resumeItemListStart
    \resumeItem{带领团队开发基于AI的旅行规划平台,自定义训练LLM模型集成于 React Native,针对用户偏好生成个性化行程}
    \resumeItem{使用 LangChain 及向量数据库实现 RAG系统,通过利用历史行程数据提升推荐效果}
    \resumeItem{采用 AWS Lambda 与 Node.js 搭建可扩展后端基础设施,实现基于微服务和 RESTful API 的高效数据处理}
    \resumeItem{领导 6 人的敏捷开发团队完成从概念到 MVP 的全流程,定期组织代码审查和技术方案讨论}
  \resumeItemListEnd

\resumeSubHeadingListEnd

\section{技能}
\resumeSubHeadingListStart
  \resumeSubItem{编程语言:}{Python, JavaScript, C/C++, C\#, HTML, Swift, MATLAB}
  \resumeSubItem{框架与库:}{React Native, React.js, Node.js, Django, LangChain, OpenCV, YOLO, PyTorch}
  \resumeSubItem{云与 DevOps:}{AWS(Lambda), Google Cloud Platform, Firebase, Git, CI/CD}
  \resumeSubItem{开发工具:}{VS Code, Xcode, Unity, Fusion 360, Blender, AutoCAD}
\resumeSubHeadingListEnd

\end{document}
